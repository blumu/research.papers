\documentclass{article}
\def\ord{\sf ord}

\begin{document}

\section{Rules design explanation}

\subsection{remark 1}

\textbf{Question}:

Why not generalising $\vdash^-$ by introducing $\vdash^{-k}$ for
$k>1$ with the following meaning: $ \Gamma \vdash^{-k} M : A$ when
variable in $\Gamma$ have orders at least $\ord(M)-k$?
\\


\textbf{Answer}: The only rule that really exploit the judgment
$\vdash^-$ is the weakening rule. We can see that this rule is only
necessary for $k=1$ and therefore there is no need for
generalization.



Consider $M : \overline{A} | B$. Because of homogeneity of the
partitioned type $\overline{A} | B$ we have $\ord(M) = 1 + \ord(A)$.

Note that there are only two cases where we need to weaken a
judgment $\Gamma \vdash M : \overline{A} | B$ to $\Gamma, x: C
\vdash^- M : \overline{A} | B$ while forming an unsafe term. In both
cases, the hope is that eventually we will form a safe term. The two
cases are:

\begin{itemize}
\item
 we want to form a safe term by abstracting a variable. The requirement here is to respect type homogeneity
 therefore we need to have $\ord(x) \geq \ord(A)$.
Hence the weakening should allow the case when $\ord(x) = \ord(A) =
\ord(M) - 1$.

\item
 we want to form a safe term by apply another term $N$ to $M$ such
that $N$ has a free variable that do not appear $M$. This weakening
can always be postpone until we really need it : just before
instancing the (App) rule with $M$ and $N$ as the premises.

The only way to recover the unsafety of $\Gamma, x: C \vdash^- M :
\overline{A} | B$ is to applied the safe term $N$ to it (in that way
the unsafe term $M$ does not appear at an operand position).

But since $x$ occurs freely in the safe term $\Gamma, x: C \vdash N$
we should have $\ord(x) \geq \ord(N) = \ord(A) = \ord(M) -1$
\end{itemize}

Hence the only weakening rule that we need is $\vdash^-1$.

\end{document}
