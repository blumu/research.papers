% -*- TeX -*- -*- Soft -*-
%Daemon> ini=pdflatex
%Daemon> filter=err+warn
%Daemon> custom_args="-synctex=1"

\documentclass{article}
\usepackage{amsmath, amsthm, amssymb}
\usepackage{a4wide}
\usepackage{gamesem}
\usepackage{pstring}
\usepackage{amsmath,amssymb,amsthm}

\theoremstyle{definition}
\newtheorem{definition}{Definition}[section]

\newcommand\Nodes{N}% set of nodes

\newcommand{\ghostlmd}{{\lambda\!\!\lambda}}
\newcommand{\ghostvar}{\theta}

\author{William Blum}
\title{[Draft] Traversals for ULC: On-the-fly eta-expanded traversals}

\begin{document}
\maketitle
\begin{abstract}
On-the-fly eta-expanded traversal for the untyped lambda calculus.
(Adapted from the definition used for simply-typed lambda calculus in my thesis \cite{BlumPhd}.)
\end{abstract}


\section{Definition}

For an untyped lambda term $M$ we defined its \defname{computation tree} $\tau(M)$ the same way as for the STLC
except that we don't perform eta-long expansion. Consecutive lambda abstractions are still merged into a single nodes in the tree so as to maintain the alternation between lambda nodes (at odd level) and variable nodes (at even level).
Let $\Nodes$ be the set of nodes of the computation tree. We write $\Nodes_{\sf var}$ for the set of variable nodes, $\Nodes_{\sf \lambda}$ for lambda nodes.

We defined the \defname{arity} of a node as follows: the arity of a lambda node $\lambda x_1 \cdots x_k$ for $k\geq 0$ is denoted by $|\lambda x_1 \cdots x_k|$ and is defined as $k$; the arity of a variable node $x$, denoted $|x|$ is the number of children of $x$ in the computation tree; the arity of a $@$ node is the number of its children nodes minus 1.

The set $\travset(M)$ of \defname{traversals} over an untyped lambda term $\tau(M)$ is a sequence of elements in $\Nodes + \{ \ghostlmd, \ghostvar \}$ with associated pointers. It is defined by induction over the rules of Table \ref{tab:trav_rules}.
Elements in $N$ represent occurrences of nodes from the original term tree while $\ghostlmd$ and $\ghostvar$ are placeholders representing eta-expanded ``ghost'' lambda nodes and ``ghost'' variable nodes respectively.
By convention ghost variables and lambda nodes are given arity $0$.

A traversal that cannot be extended by any rule is said to be \emph{maximal}.

Note that those rules closely match those of \cite{BlumPhd} for STLC. In the present setting we don't have interpreted constants so we removed the rules $(Value)$ and $(InputValue)$ which are not needed.

We now fix an untyped term and abbreviated its set of traversals as just $\travset$.

\begin{FramedTable}
\noindent {\bf Structural rules}
\begin{itemize}[]
\item\rulenamet{Root} The singleton sequence $r$ is in $\travset$ where $r$ is the root of the tree.
\end{itemize}

\begin{itemize}[]
    \item \rulenamet{Lam} If $t \cdot \lambda \overline{\xi}$ is a traversal then so is
        $t \cdot \lambda \overline{\xi} \cdot n$ where $n$
        denotes $\lambda \overline{\xi}$'s child in $\tau(M)$ and its justified is defined as follows:
        \begin{compactitem}
            \item If $n$ is an $@$ node then it has no justifier;
            \item if  $n \in \Nodes_{\sf var}$ but not in $\Nodes_{\sf fv}$ then it points to the only occurrence of its binder in
            $\pview{t\cdot \lambda \overline{\xi}}$;
            \item if  $n \in \Nodes_{\sf fv}$ then it points
            to the only occurrence of the root in
            $\pview{t \cdot \lambda \overline{\xi}}$.
        \end{compactitem}
            (As in STLC, it can be shown that P-views  are paths in the (possibly eta-expanded) tree thus $n$'s enabler occurs exactly once in the P-view.)
    
    \item \rulenamet{Lam^\ghostlmd} Suppose
  $\Pstr[0.5cm]{ t \cdot(l){l} \cdot
(v){v} \cdot \ldots \cdot 
(gl-v,40:i){\ghostlmd} \in\travset}$ for some traversal prefix $t$, lambda node $l \in \Nodes + \{ \ghostlmd \}$ and variable node $v \in \Nodes + \{ \ghostvar\}$ then
    
    
$$\Pstr[0.5cm]{ t \cdot(l){l} \cdot
(v){v} 
\cdot \ldots \cdot 
(gl-v,40:i){\ghostlmd}\cdot (al-l,40:j){\ghostvar}
     \in\travset}$$

     where $j = |l| + i - |v|$. The placeholder $\ghostvar$ represents an occurrence of the $j$th variable bound by lambda node $l$ if the sub-term at node $l$ were eta-expanded $i-|v|$ times.


    \item \rulenamet{App} If $t \cdot @$ is a traversal then so is \Pstr[0.4cm]{t \cdot (m) @  \cdot (n-m,40:0) n}.
\end{itemize}

\emph{\bf Data - Input-variable rules}


\begin{itemize}[]
\item \rulenamet{InputVar} If $t$ is a traversal whose last occurrence is a variable hereditarily justified by the root and $x$ is an occurrence of a variable node in $\oview{t}$ then
so is $t \cdot n$ for every child $\lambda$-node $n$ of $x$, $n$
pointing to $x$.
\end{itemize}

\emph{\bf Program - Copy-cat rules \underline{with on-the-fly eta-expansion}}

Suppose \Pstr[0.5cm]{t \cdot(l){l} \cdot (n){n} \cdot (lx){\lambda \overline{x} }
    \ldots (x-lx,50:i){x_i} \in \travset} where $i>0$, $x_i$ is a variable in $\Nodes + \{ \ghostvar\}$ hereditarily justified by an $@$-node; $n \in \Nodes + \{ \ghostvar \}$ is either a (ghost) variable or application node; and $l$ is a lambda node in $\Nodes + \{\ghostlmd\}$. Then:

\begin{itemize}
  \item \rulenamet{Var} (No expansion) If $i \leq |n|$ then
  $$\Pstr[0.5cm]{ t \cdot(l){l} \cdot
(n){n} \cdot (lx){\lambda \overline{x}}  \ldots (x-lx,30:i){x_i}
    \cdot (letai-n,40:i){\lambda \overline{\eta_i}}
     \in\travset}$$
where $\lambda \overline{\eta_i} \in N$ is the $i$th child of $n$ (which is necessarily a node of the tree since $|n|\geq|i|>0$).

\item \rulenamet{Var^\eta} (Eta-Expansion) If $i > |n|$ then
  $$\Pstr[0.5cm]{ t \cdot(l){l} \cdot
(n){n} \cdot (lx){\lambda \overline{x}}  \ldots (x-lx,30:i){x_i}
    \cdot (letai-n,40:i){\ghostlmd}
     \in\travset}$$    
     where the ghost node $\ghostlmd$ represents the $i$th ghost child of $n$.
\end{itemize}

\caption{Traversal rules for the untyped lambda calculus.}
 \label{tab:trav_rules}
\end{FramedTable}


\bibliographystyle{abbrv}
\bibliography{ulctrav}

\end{document}
